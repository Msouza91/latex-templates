% xelatex
\documentclass[letterpaper,
		%twocolumn,
		10pt]{article}
\usepackage[utf8]{inputenc}
\usepackage{metalogo}
\usepackage{xifthen}
\usepackage[colorlinks=true,urlcolor=Blue]{hyperref}
\usepackage{graphicx}
\usepackage{fontspec}
\usepackage[T1]{fontenc}
\usepackage[dvipsnames]{xcolor}
\usepackage{titlesec}
\usepackage[margin=1in]{geometry}
\usepackage{titling}
\newfontfamily\cfont{Noto Sans CJK SC}
\usepackage{libertine}

% Macro to allow image links in XeLaTeX
\ifxetex
  \usepackage{letltxmacro}
  \setlength{\XeTeXLinkMargin}{1pt}
  \LetLtxMacro\SavedIncludeGraphics\includegraphics
  \def\includegraphics#1#{% #1 catches optional stuff (star/opt. arg.)
    \IncludeGraphicsAux{#1}%
  }%
  \newcommand*{\IncludeGraphicsAux}[2]{%
    \XeTeXLinkBox{%
      \SavedIncludeGraphics#1{#2}%
    }%
  }%
\fi
%%%%%%%

% Bold contents of a link
\let\oldhref\href
\renewcommand{\href}[3][blue]{\oldhref{#2}{\color{#1}{#3}}}

% Your name goes here:
\author{Marcos Souza}

% Update date set to last compile:
\date{\today}

% Custom title command.
\renewcommand{\maketitle}{
	\hspace{.25\textwidth}
	\begin{minipage}[t]{.5\textwidth}
\par{\centering{\Huge  \bfseries{\theauthor}}\par}
	\end{minipage}
	\begin{minipage}[t]{.25\textwidth}
{\footnotesize\hfill{}\color{gray}

\hfill{}(Last updated \thedate.)
}
	\end{minipage}
}



% Setting the font I want:
\renewcommand{\familydefault}{\sfdefault}
\usepackage{sqrcaps}

% Making the \entry command
\newcommand{\entry}[4]{
\ifthenelse{\isempty{#3}}
{\slimentry{#1}{#2}}{

\begin{minipage}[t]{.15\linewidth}
\hfill \textsc{#1}
\end{minipage}
\hfill\vline\hfill
\begin{minipage}[t]{.80\linewidth}
{\bf#2}\\\textit{#3} \footnotesize{#4}
\end{minipage}\\
\vspace{.2cm}
}}

\newcommand{\slimentry}[2]{

\begin{minipage}[t]{.15\linewidth}
\hfill \textsc{#1}
\end{minipage}
\hfill\vline\hfill
\begin{minipage}[t]{.80\linewidth}
#2
\end{minipage}\\
\vspace{.25cm}
}% end \entry command definition

% Some macros because I'm lazy:
\newcommand{\uga}{University of Georgia}
\newcommand{\gsu}{Georgia State University}
\newcommand{\ua}{University of Arizona}

% Macros for people's names including link to their websites
\newcommand{\tgb}{\href{http://coglanglab.com}{Tom Bever}}
\newcommand{\mas}{\href{http://dingo.sbs.arizona.edu/~massimo/}{Massimo Piattelli-Palmarini}}
\newcommand{\rob}{\href{https://rhenderson.net/}{Robert Henderson}}
\newcommand{\mike}{\href{http://www.u.arizona.edu/~hammond/}{Mike Hammond}}
\newcommand{\simin}{\href{http://www.u.arizona.edu/~karimi/}{Simin Karimi}}
\newcommand{\heidi}{\href{http://heidiharley.com/}{Heidi Harley}}
\newcommand{\amy}{\href{https://linguistics.arizona.edu/user/amy-fountain}{Amy Fountain}}
\newcommand{\vera}{\href{https://www.gsstudies.uga.edu/people/vera-lee-schoenfeld}{Vera Lee-Schoenfeld}}
\newcommand{\tim}{\href{http://www.rom.uga.edu/directory/timothy-gupton}{Tim Gupton}}
\newcommand{\pilar}{\href{http://www.rom.uga.edu/directory/pilar-chamorro}{Pilar Chamorro}}
\newcommand{\jenni}{\href{https://www.jennimariapalomaki.com/}{Jennimaria Palomäki}}

\let\lineheight\baselineskip

% Link images
\newcommand{\pdf}{\includegraphics[height=.85em]{cv/pdf.png}}
\newcommand{\yt}{\includegraphics[height=.85em]{cv/yt.png}}
\newcommand{\gh}{\includegraphics[height=.85em]{cv/gh.png}}
\newcommand{\www}{\includegraphics[height=.85em]{cv/www.png}}
\newcommand{\email}{\includegraphics[height=.85em]{cv/email.png}}

% Custom section spacing and formatting
\titleformat{\part}{\Huge\scshape\filcenter}{}{1em}{}
\titleformat{\section}{\Large\bf\raggedright}{}{1em}{}[{\titlerule[2pt]}]
\titlespacing{\section}{0pt}{3pt}{7pt}
\titleformat{\subsection}{\large\bfseries\centering}{}{0em}{\underline}%[\rule{3cm}{.2pt}]
\titlespacing{\subsection}{0pt}{7pt}{7pt}

% No indentation
\setlength{\parindent}{0in}

\begin{document}

\maketitle

\section{Basic Info}

\begin{minipage}[t]{.5\linewidth}
	\begin{tabular}{rp{.75\linewidth}}
		\baselineskip=20pt
		\email{} : & \href{mailto:me@msouza.tech}{me@msouza.tech}        \\
		\www{} :   & \href{https://www.msouza.tech}{https://msouza.tech}
	\end{tabular}
\end{minipage}
\begin{minipage}[t]{.5\linewidth}
	\begin{tabular}{rl}
		\gh{} : & \href{http://github.com/msouza91}{github.com/msouza91} \\
	\end{tabular}
\end{minipage}

\begin{itemize}
	\item Azure DevOps Expert Certified.

	      Credential ID: 5BA84FF1F7002185
	      Certification number: 8F43A0-L059E6

	      For more information, go here: \href{https://learn.microsoft.com/en-us/credentials/certifications/devops-engineer/}{https://learn.microsoft.com/en-us/credentials/certifications/devops-engineer/}.
	\item Openshipt Development 101 - \href{https://rol.redhat.com/rol/api/certificates/attendance/uuid/197e4993-1a79-4ee1-9688-e06893134f82}{Certificate}
\end{itemize}

\section{Experience}

\entry{2016--2017}
{I.T Support}
{Municipal Secretariat for Human Rights, Fortaleza, Ceará - }
{Level 1 Support, responsible for the maintenance of the computer network, printers and PCs.}

\entry{2017--2022}
{Site Reliability Engineer}
{Foundation for Science, Technology and Innovation (CITINOVA), Fortaleza, Ceará - }
{Management of the infrastructure of the institution's servers, responsible for the maintenance of the network, servers and services.}

\entry{2022--Now}
{Devops Engineer}
{Iteris Consultoria e Software, São Paulo, São Paulo - }
{designing and maintaining CI/CD pipelines using GitHub Actions and Azure DevOps, creating Infrastructure as Code configurations with Terraform and Azure Bicep for Azure and AWS deployments, deploying applications in virtualized and containerized environments like ECS and Kubernetes, developing Helm charts, and writing Bash and PowerShell scripts for pipeline and environment customization.}

\section{Education}

\entry{2017--2022}
{Sysem Analysis and Development}
{Centro Universitário Estácio do Ceará, Fortaleza, Ceará}
{}

\section{Languages}

\entry{Human}
{\href{https://www.efset.org/cert/aXzYTL}{ English }, Portuguese}
{}
{}

\entry{Machine}
{Python, bash/shell, some superficial knowledge of C and Go; markup languages including {\LaTeX}/{\XeTeX}, Markdown, HTML, CSS.}
{}
{}

\section{Tools I Use}

\subsection{Usual Workflow}

I use a \textbf{vim}-based setup in a tiling window manager (\textbf{AwesomeWm}). I compile documents using \textbf{Markdown} or \textbf{\LaTeX}, and \textbf{biber} for references. I prefer to do multimedia manipulation in the terminal with tools like \textbf{imagemagick} and \textbf{ffmpeg} for extensibility's sake.
I've run Microsoft, MacOS and GNU/Linux systems (both Debian and Arch-based varieties).

\subsection{Programs I'm Familiar With}

tmux, ssh, GIMP, pandoc, Jupyter. I've managed websites manually via ssh and vim using Markdown/Go and with tools such as Github Pages, and Azure Static Websites.

\section{Public Code and Scripts}

\entry{2018}
{mutt and offlineIMAP wizard \href{https://github.com/lukesmithxyz/mutt-wizard}{\gh}}
{A fully-featured configuration tool.}
{A tool for automatically configuring mutt, offlineIMAP, notmuch and other parts of a full email system with automatic detection of IMAP and SMTP protocols and automatic password encryption}

\entry{2017--2018}
{shortcut-sync \href{https://github.com/lukesmithxyz/shortcut-sync}{\gh}}
{A small system for synchronizing configs between bash and other shells, ranger and qutebrowser.
}
{}

\entry{In Progress}
{Corpus Latinum Lucæ \href{https://github.com/LukeSmithxyz/corpus-latinum}{\gh}}
{A planned linguistic corpus of the Latin language, to be searchable by regexes and grammatical category, with an interface written in Python.}
{As of now I've constructed the parser to generate the corpus from raw text files.}

\entry{2017--Now}
{Luke's Auto-Rice Bootstrapping Scripts (LARBS) \href{https://larbs.xyz}{\www} \href{https://github.com/LukeSmithxyz/larbs}{\gh}}
{A dynamic installer for an i3wm Arch Linux distribution}
{}

\entry{2016--Now}
{Voidrice \href{https://github.com/LukeSmithxyz/voidrice}{\gh}}
{Linux dotfiles}
{A set of GNU/Linux dotfiles that I popularized on YouTube, aiming at creating a powerful, optimized, all-purpose and lightweight general computing environment. Innovated dynammically configured and synced rc files.}

\entry{2015}
{http://LukeSmith.xyz \href{http://LukeSmith.xyz}{\www}}
{My website}
{Written from scratch. Modern traditional feel. I generate the website offline with PHP and upload a static version of only HTML/CSS.}

\entry{2016--2017}
{/Comfy/ Arch}
{Automatic ricing tool}
{A web-based deployment system for configuration files which focuses on enabling novice Linux users to get immersed quickly in advanced ricing environments. No longer actively maintained as it has been largely replaced by LARBS (above).}

\entry{2013}
{Vulgarizer}
{Sound Change Simulator}
{A string-manipulation paradigm written in Python for simulating phonetic and phonological change over time. The original implementation focused on modeling changes between Latin and Spanish and then other Romance languages, but if I continue the project, my hope is to generalize this.}

%\subsection{Other}

%My build of the the suckless \href{https://github.com/LukeSmithxyz/st}{simple terminal (st) \gh} with added homerow binds and small improvements.

\section{Online Tutorials}

I produce screencasts and produce other videos for public use as tutorials on YouTube on my channel. \href{https://youtube.com/c/LukeSmithxyz}{\yt}
As of January 2018, over 13,000 subscribers, with 1.4 millions views on over 100 different videos. All of these series below are ongoing.

\entry{R / Statistics \href{https://www.youtube.com/playlist?list=PL-p5XmQHB_JQlsPHtcxNWpvDudyoWB2kH}{\yt}}
{\footnotesize
	\href{https://www.youtube.com/watch?v=WlCWQrKQQI4}{[1]: Arithmetic, Variables and Vectors},
	\href{https://www.youtube.com/watch?v=pgRvPNo6HaY}{[2]: Basic Statistical Functions and Logic},
	\href{https://www.youtube.com/watch?v=DDzVtu24dbk}{[3]: Dataframes, Subsetting and Ifelse}
}
{}{}

\entry{{\LaTeX}
	\href{https://www.youtube.com/playlist?list=PL-p5XmQHB_JSQvW8_mhBdcwEyxdVX0c1T}{\yt}}
{\footnotesize
	\href{https://youtube.com/watch?v=NwnYHoNtfJ0}{[0]: Installation}
	\href{https://youtube.com/watch?v=mfRmmZ_84Mw}{[1]: Compiling, Titles, Sections, Formatting, Syntax}
	\href{https://youtube.com/watch?v=25LExaNtdF0}{[2]: Labels, References and Lists}
	\href{https://youtube.com/watch?v=46piog3Fzp4}{[3]: Bibliographies}
	\href{https://youtube.com/watch?v=VjsX4tznW40}{[4]: Résumé-making Part 1}
	\href{https://youtube.com/watch?v=o5-BZ7JmYWk}{[5]: Résumé-making Part 2}
	\href{https://youtube.com/watch?v=zgThRPjy-vw}{[6]: Images, Figures}
	\href{https://youtube.com/watch?v=zEjBCQhND2c}{[7]: Beamer Slideshow Presentations}
	\href{https://youtube.com/watch?v=rvgP7IMeUn8}{[8]: Macros}
}
{}{}

\entry{vim
	\href{https://www.youtube.com/playlist?list=PL-p5XmQHB_JSTaEPygu1DZjuFfb704Uv7}{\yt}
}
{\footnotesize
	\href{https://youtube.com/watch?v=wRFEBw02aT8}{Macros}---
	\href{https://youtube.com/watch?v=yNOkCYuPt3E}{Workflow with LaTeX}---
	\href{https://youtube.com/watch?v=jUfw7aHD_xY}{Basic Tips After Vimtutor}---
	\href{https://youtube.com/watch?v=K_8_gazN7h0}{SC-IM - Vim-based Terminal Spreadsheet Editor}---
	\href{https://youtube.com/watch?v=Q4I_Ft-VLAg}{Custom code snippets, for IDE functionality}---
	\href{https://youtube.com/watch?v=GqoJQft5R2E}{Vi-mode in bash}---
	\href{https://youtube.com/watch?v=ez1XBUqbS68}{Spell-checking and Dictionaries}---
	\href{https://youtube.com/watch?v=NzD2UdQl5Gc}{How vim Makes my Daily Life Easier}
}{}{}

\entry{Linux}
{Videos on hacking and modifying Linux graphical environments particularly on i3-gaps, including customization, window management, program-shortcutting, optimization.
	\href{https://www.youtube.com/playlist?list=PL-p5XmQHB_JTcMSvPmXMzNe7ZPMxEx_Oz}{\yt}}
{}
{}

\section{Service}
\entry{Soon}
{Coyote Papers Editor for the 2018 Arizona Linguistics Circle}
{\ua}
{}

\entry{2017--Now}
{Manager for Tom Bever's Language and Cognition Lab site \href{http://coglanglab.com}{\www}}
{\ua}
{\href{http://coglanglab.com}{http://coglanglab.com}}

\entry{2017}
{Arizona Linguistic Circle Peer-Reviewer}
{\ua}
{Judged and gave feedback for scholarly articled submitted for the Arizona Linguistics Circle.}

\entry{2015--2016}
{Indo-European Reading Group Director}
{\ua}
{Covering general historical linguistics, and the particulars of PIE morphology, phonology and development. Group materials located at \href{http://www.lukesmith.xyz/pie}{lukesmith.xyz/pie \www}.}

\entry{2014--2016}
{LSUGA Website Manager}
{\uga}
{Created and managed the LSUGA website, \href{http://www.lsuga.com}{http://www.lsuga.com}. (Now apparently reformatted.)}

\entry{2014--2015}
{LSUGA Interdisciplinary Conference in Linguistics Committee Member}
{\uga}
{Managed email, reviewed papers and did audio and video for conference.}

\entry{2014--2015}
{Graduate Student Mentor}
{\uga}
{}

\entry{2014}
{Socio-Paths Sociolinguistics Reading Group Co-director}
{\uga}

\entry{2014}
{Typology Reading Group Director}
{\uga}
{Covered classical Greenbergian typology, particularly with relevance to historical linguistics.}


\section{Writings}

Section incomplete. Only contains major degree requirements now. To be updated. More minor squibs of mine can be found at \href{https://lukesmith.xyz/linguistics}{https://lukesmith.xyz/linguistics \www}.

\entry{SOON}
{Unnamed dissertation}
{Dissertation, \ua}
{
	Probable committee?: \tgb, \mike, \mas, \rob.
}

\entry{2017}
{Scope Without Syntax: A Game Theoretic Approach
	\href{https://lukesmith.xyz/qp2.pdf}{\pdf}
	\href{https://github.com/LukeSmithxyz/scope-without-syntax}{\gh}
}
{Second qualifying paper, \ua}{
	Committee: \rob, \mas, \tgb, \mike.
}

\entry{2017}
{Syntax Without Syntax
	\href{https://lukesmith.xyz/qp1.pdf}{\pdf}
	\href{https://www.youtube.com/watch?v=fEY5qkgH3fo}{\yt}
	\href{https://github.com/lukesmithxyz/syntax-without-syntax}{\gh}
}{First qualifying paper, \ua}{
	Committee: \mike, \simin, \heidi.
}

\entry{2015}
{External Possession and the Undisentanglability of Syntax and Semantics \href{https://lukesmith.xyz/thesis.pdf}{\pdf}}
{Master's thesis, \uga}{
	Committee: \vera, \tim, \pilar.
}

\section{Hobbies}

Classical languages, human evolution and prehistory, free (i.e. libre) software, survivalism, cybernetics, medieval thought, Rhaeto-Romance poetry. I run a web forum at \href{https://forum.lukesmith.xyz}{forum.lukesmith.xyz}.

\section{References}

\begin{itemize}
	\item Email me \href{mailto:luke@lukesmith.xyz}{\email} with your interests in me and I'll refer you to someone who can vouch for me or defame me, depending on what you want.

	\item I upload all of my teaching evaluations to \href{https://lukesmith.xyz/classes\#evals}{https://lukesmith.xyz/classes\#evals \www} and they can be read there.
\end{itemize}


\end{document}
